%
% Präambel
% ===========================================================================
%

% Schriftgröße, Layout, Papierformat, Art des Dokumentes
\documentclass[
		a4paper,			% A4 Papier
		11pt,				% Standard-Schriftgröße
		twoside,			% 1-seitig / 2-seitig (oneside/twoside)
		openany,			% Kapitel können sowohl links als auch rechts beginnen
		halfparskip,		% Europäischer Satz mit halben Zeilenabstand zwischen Absätzen
		normalheadings		% Überschriften in normaler Schriftgröße (wird unten überschrieben)
		%fleqn			% Formeln linksbündig
	]{scrreprt}

% Einstellungen der Seitenränder
\usepackage[left=3cm, right=2cm, top=1.3cm, bottom=1.3cm,
		includeheadfoot, headsep=0.7cm, footskip=1cm
	]{geometry}

% Umlaute, Deutsche Trennungen, Anführungsstriche und mehr:
\usepackage[utf8]{inputenc}
\usepackage[T1]{fontenc}	% u.a. spitze Klammern
\usepackage[ngerman]{babel}
\usepackage[babel]{csquotes}


%
% ===========================================================================
%

% Meta-Informationen
\newcommand{\mytitle}{SiDiff 2.0: Vergleichsfunktionen}
\newcommand{\mydate}{\today}
\newcommand{\myauthor}{Pit Pietsch, Timo Kehrer}
\newcommand{\mypdftitle}{SiDiff 2.0: Vergleichsfunktionen}
\newcommand{\mypdfauthor}{Universität Siegen, Praktische Informatik}
\newcommand{\mypdfsubject}{Dokumentation}
\newcommand{\mypdfkeywords}{}

\newcommand{\myorg}{
		Universität Siegen\\
		Fachbereich 12\\
		Elektrotechnik und Informatik\\
		\textbf{Fachgruppe Praktische Informatik}\\
		Hölderlinstr. 3\\
		D-57068 Siegen
}

\newcommand{\myfootinner}{\mytitle}
\newcommand{\myfootouter}{Universität Siegen, Praktische Informatik}

\setcounter{tocdepth}{2}


%
% ===========================================================================
%

% Schusterjungen und Hurenkinder vermeiden
\clubpenalty = 10000
\widowpenalty = 10000

% Unterstreichungen
\usepackage[normalem]{ulem}	% (\uline \uuline \dotuline \dashuline)
\def\dotuline{\bgroup
  \ifdim\ULdepth=\maxdimen
   \settodepth\ULdepth{(j}\advance\ULdepth.2pt\fi
  \markoverwith{\begingroup
  \advance\ULdepth0.08ex
  \lower\ULdepth\hbox{\kern.1em .\kern.005em}%
  \endgroup}\ULon}
\def\dashuline{\bgroup
  \ifdim\ULdepth=\maxdimen
   \settodepth\ULdepth{(j}\advance\ULdepth.2pt\fi
  \markoverwith{\kern.15em
  \vtop{\kern\ULdepth \hrule width .3em}%
  \kern.15em}\ULon}


% Zeilenabstand
% (\singlespacing \onehalfspacing \doublespacing \setstretch{1})
\usepackage{setspace}
\newcommand{\mylinespacing}{\singlespacing}
\mylinespacing

% Type 1 Fonts für bessere Darstellung in PDF verwenden
%\usepackage{mathptmx}				% Times (\rmdefault) und Mathematikschrift
\usepackage[T1]{eulervm} 			% Euler u.a. als Mathematikschrift
%\usepackage{txfonts}				% Times (\rmdefault), Adobe Helvetica (\sfdefault), TXTT (\ttdefault) etc.
%\usepackage[scaled=.92]{helvet}		% Helvetica (\sfdefault)
%\usepackage{courier}				% Courier (\ttdefault)
%\usepackage[scaled]{beramono}		% Bera Mono (\ttdefault)
\renewcommand*{\ttdefault}{txtt}		% TXTT (\ttdefault)
%\renewcommand{\familydefault}{\sfdefault}
\usepackage{cancel} 		% Kürzen erlauben
\usepackage{underscore}		% Underscore in Typewriter erlauben (\textunderscore{})
\newcommand{\degree}{^{\circ}}


% Zahlenausgabe (Tausendertrennzeichen)		\numprint{}
\usepackage{numprint}

% Mathematische Symbole aus dem AMS Paket
\usepackage{amsmath}
\usepackage{amssymb}

% Package für Farben im PDF
\usepackage{color}

% Font-Kommandos:
% \sffamily		\textsf{}		serifenlos (Arial o.ä.)
% \rmfamily		\textrm{}		Romanschriftart (Times)
% \ttfamily		\texttt{}		Schreibmaschinenschrift (Courier o.ä.)
% 
% \mdseries		\textmd{}		normale (mittlere) Stärke
% \bfseries		\textbf{}		Fettdruck
%
% \itshape		\textit{}		Kursivschrift
% \slshape		\textsl{}		geneigte Schrift
% \scshape		\textsc{}		Kapitälchenschrift
% \upshape		\textup{}		Normale Schrift (Zurückschaltung)

% Spezielle Schrift im Koma-Script setzen
\setkomafont{chapter}{\Large\sffamily\upshape\bfseries}
\setkomafont{section}{\large\sffamily\upshape\bfseries}
\setkomafont{subsection}{\normalfont\sffamily\itshape\bfseries}
\setkomafont{subsubsection}{\small\sffamily\itshape\bfseries}
\setkomafont{sectioning}{\normalfont\rmfamily\upshape\mdseries}
\setkomafont{caption}{\footnotesize\sffamily\upshape\mdseries}
\setkomafont{captionlabel}{\footnotesize\sffamily\upshape\bfseries}
\setkomafont{descriptionlabel}{\small\sffamily\upshape\mdseries}

% Absätze
\newcommand{\setparindent}[1]{\parindent#1}		% Einrückung der ersten Zeile eines Absatzes
\newcommand{\setparskip}[1]{\parskip#1}			% Abstand zwischen zwei Absätzen
\newcommand{\setmyparindent}{\setparindent{0pt}}
\newcommand{\setmyparskip}{\setparskip{6pt}}
\setmyparindent
\setmyparskip

% Kapitel anpassen (Seitenumbruch)
\makeatletter
	\renewcommand\chapter{
			\if@openright\cleardoublepage\else\clearpage\fi
			\thispagestyle{plain}%
			\global\@topnum\z@
			\@afterindenttrue
			\secdef\@chapter\@schapter
	}
	\def\@chapter[#1]#2{\ifnum \c@secnumdepth >\m@ne
	      \refstepcounter{chapter}%
	      \@maybeautodot\thechapter
	      \typeout{\@chapapp\space\thechapter.}%
	      \addcontentsline{toc}{chapter}%
	      {\protect\numberline{\thechapter}#1}%
	    \else
	      \addcontentsline{toc}{chapter}{#1}%
	    \fi
	    \chaptermark{#1}%
	    \addtocontents{lof}{\protect\addvspace{10\p@}}%
	    \addtocontents{lot}{\protect\addvspace{10\p@}}%
	    \addtocontents{lol}{\protect\addvspace{10\p@}}%			% INSERTED
	    \@ifundefined{float@addtolists}{}{%
	      \float@addtolists{\protect\addvspace{10\p@}}%
	    }%
	    \if@twocolumn
	      \if@at@twocolumn
	        \@makechapterhead{#2}%
	      \else
	        \@topnewpage[\@makechapterhead{#2}]%
	      \fi
	    \else
	      \@makechapterhead{#2}%
	    \@afterheading
	  \fi
	}
	\renewcommand*{\@makechapterhead}[1]{%
	  \use@chapter@o@preamble
	  \@@makechapterhead{#1}%
	  \use@preamble{chapter@u}\nobreak
	}
	\renewcommand*{\@@makechapterhead}[1]{\chapterheadstartvskip
	  {\normalfont\sectfont\size@chapter
	    \setlength{\parindent}{\z@}\setlength{\parfillskip}{\fill}%
	    \if@chapterprefix
	      \let\@tempa\raggedsection
	    \else
	      \let\@tempa\@hangfrom
	    \fi
	    \@tempa{\ifnum \c@secnumdepth >\m@ne%
	        \chapterformat
	      \fi
	    }%
	    \if@chapterprefix\par\nobreak\vskip.5\baselineskip\fi
	    {\raggedsection \interlinepenalty \@M #1\par}}%
	  \nobreak\chapterheadendvskip
	}
	\renewcommand*{\@schapter}[1]{%
	  \if@twocolumn
	    \if@at@twocolumn
	      \@makeschapterhead{#1}%
	    \else
	      \@topnewpage[\@makeschapterhead{#1}]%
	    \fi
	  \else
	    \@makeschapterhead{#1}%
	    \@afterheading
	  \fi
	}
	\renewcommand*{\@@makeschapterhead}[1]{%
	  \chapterheadstartvskip%
	  {\normalfont\sectfont\size@chapter
	    \setlength{\parindent}{\z@}\setlength{\parfillskip}{\fill}%
	    \raggedsection \interlinepenalty \@M #1\par}%
	  \nobreak\chapterheadendvskip%
	}
	\renewcommand*{\@makeschapterhead}[1]{%
	  \use@chapter@o@preamble
	  \@@makeschapterhead{#1}%
	  \use@preamble{chapter@u}\nobreak
	}
\makeatother

% Formatierung von Gliederungen
\usepackage{titlesec}
\titleformat{\chapter}{\usekomafont{chapter}}{\thechapter}{2ex}{} 		% Marke, Schrift, Text, Abstand Prefix/Text
\titleformat{\section}{\usekomafont{section}}{\thesection}{2ex}{} 		% Marke, Schrift, Text, Abstand Prefix/Text
\titleformat{\subsection}{\usekomafont{subsection}}{\thesubsection}{2ex}{}	% Marke, Schrift, Text, Abstand Prefix/Text
\titleformat{\subsubsection}{\usekomafont{subsubsection}}{\thesubsubsection}{2ex}{}	% Marke, Schrift, Text, Abstand Prefix/Text
\titlespacing*{\chapter}{0pt}{-15pt}{15pt}	% Abstand: zum linken Rand, nach oben, nach unten
\titlespacing*{\section}{0pt}{10pt}{5pt}		% Abstand: zum linken Rand, nach oben, nach unten
\titlespacing*{\subsection}{0pt}{5pt}{-3pt}	% Abstand: zum linken Rand, nach oben, nach unten
\titlespacing*{\subsubsection}{0pt}{5pt}{-3pt}	% Abstand: zum linken Rand, nach oben, nach unten

% Paket um Textteile drehen zu können
%\usepackage{rotating}

% Erweiterte Tabellen
\usepackage{tabularx}
\usepackage{tabulary}
\usepackage{multirow}	% Multirow erlauben
%\usepackage{arydshln}	% Dashed Lines
\usepackage{booktabs}	% Rules

% Berechnungen zulassen
\usepackage{calc}

% Paket um auch Titelnamen referenzieren zu können (\nameref -> \label muss nach \chapter etc. und \caption stehen)
\usepackage{nameref}
\newcommand{\structgetref}[1]{\ref{#1} \glq \nameref{#1}\grq}
\newcommand{\structgetfullref}[1]{Abschnitt \ref{#1} \glq \nameref{#1}\grq}
\newcommand{\structgetfullrefgen}[1]{Abschnitts \ref{#1} \glq \nameref{#1}\grq}
\newcommand{\structgetfullrefchap}[1]{Kapitel \ref{#1} \glq \nameref{#1}\grq}
\newcommand{\structgetfullrefchapgen}[1]{Kapitels \ref{#1} \glq \nameref{#1}\grq}

% Kopf- und Fußzeile (siehe auch Geometry)
\usepackage[
	headsepline,
	footsepline,
	plainheadsepline,
	plainfootsepline
	]{scrpage2}
\pagestyle{scrheadings}
\renewcommand{\headfont}{\footnotesize\sffamily\upshape\mdseries} % Headings
\renewcommand{\pnumfont}{\footnotesize\sffamily\upshape\mdseries} % Seitenzahlen
	% Formate: [verschiebung] (wenn nicht angeg. autom. zentriert) {breite}
%\setheadwidth[0pt]{textwithmarginpar}	
%\setfootwidth[0pt]{textwithmarginpar}
	% Formate: [länge]{dicke}
%\setheadsepline[17cm]{.1pt} 
%\setfootsepline[17cm]{.1pt}
\renewcommand*{\chapterpagestyle}{scrheadings}
\clearscrheadfoot
\newcommand{\headseplinespacetohead}{}
\newcommand{\footseplinespacetofoot}{\vspace*{-3pt}}
% Werte definieren
\newcommand{\oheadval}{\headseplinespacetohead\upshape\pagemark}
\newcommand{\iheadval}{\headseplinespacetohead\headmark}
\newcommand{\cheadval}{\headseplinespacetohead}
\newcommand{\ofootval}{\footseplinespacetofoot \myfootouter}
\newcommand{\ifootval}{\footseplinespacetofoot \myfootinner}
\newcommand{\cfootval}{\footseplinespacetofoot}
\newcommand{\hfplainoverwritten}{
	% Formate: [Plainstil] {normaler Stil}
	\ohead[\oheadval]{\oheadval}
	\ihead[\iheadval]{\iheadval}
	\chead[\cheadval]{\cheadval}
	\ofoot[\ofootval]{\ofootval}
	\ifoot[\ifootval]{\ifootval}
	\cfoot[\cfootval]{\cfootval}}
\newcommand{\hfplainempty}{
	% Formate: [Plainstil] {normaler Stil}
	\ohead[]{\oheadval}
	\ihead[]{\iheadval}
	\chead[]{\cheadval}
	\ofoot[]{\ofootval}
	\ifoot[]{\ifootval}
	\cfoot[]{\cfootval}}
\hfplainoverwritten
% Stile festlegen
\newcommand{\hfstyleauto}{\pagestyle{scrheadings}}
\newcommand{\hfstyleempty}{\pagestyle{empty}}
\hfstyleauto

% % [Plainstil] {normaler Stil}
\newcommand{\mymanualmark}[1]{\manualmark\markboth{#1}{#1}}
\newcommand{\myautomark}{\automark[chapter]{chapter}}
\myautomark
\setlength{\textheight}{24cm}

% Inhaltsverzeichnis
\makeatletter
	\renewcommand{\tableofcontents}{%
		\begingroup
			\if@twocolumn
				\@restonecoltrue\onecolumn
			\else
				\@restonecolfalse
			\fi
			\phantomsection
			\addcontentsline{toc}{chapter}{\contentsname}
			\chapter*{\contentsname\@mkboth{\contentsname}{\contentsname}}	%\toc@heading
			\setparsizes{\z@}{\z@}{\z@\@plus 1fil}\par@updaterelative
			\if@tocleft\before@starttoc{toc}\fi
			\@starttoc{toc}%
			\if@tocleft\after@starttoc{toc}\fi
			\if@restonecol\twocolumn\fi
		\endgroup
	}
\makeatother
% Für Zugriff auch im Anhang(!)
\newcounter{savedtocdepth}
\newcommand*{\settocdepth}[1]{
  \addtocontents{toc}{
    \protect\setcounter{savedtocdepth}{\protect\value{tocdepth}}%
    \protect\setcounter{tocdepth}{#1}
  }}
\newcommand*{\restoretocdepth}{
  \addtocontents{toc}{%
    \protect\setcounter{tocdepth}{\protect\value{savedtocdepth}}%
  }}

% Bibliographie
\usepackage{bibgerm}
%\usepackage{natbib}
% Bibliographie neudefinieren (Aufnahme Gliederung als Section)
\addto\captionsngerman{\renewcommand{\bibname}{Bibliographie}}
\makeatletter
	\renewenvironment{thebibliography}[1]{%
			\phantomsection
			\addcontentsline{toc}{chapter}{\bibname}
			\chapter*{\bibname\@mkboth{\bibname}{\bibname}}%
      \list{\@biblabel{\@arabic\c@enumiv}}%
           {\settowidth\labelwidth{\@biblabel{#1}}%
            \leftmargin\labelwidth
            \advance\leftmargin\labelsep
            \@openbib@code
            \usecounter{enumiv}%
            \let\p@enumiv\@empty
            \renewcommand\theenumiv{\@arabic\c@enumiv}}%
      \sloppy
      \clubpenalty4000
      \@clubpenalty \clubpenalty
      \widowpenalty4000%
      \sfcode`\.\@m}
     {\def\@noitemerr
       {\@latex@warning{Empty 'thebibliography' environment}}%
      \endlist}
\makeatother

% Paket für die Indexerstellung
%\usepackage{makeidx}

% Darstellung des Glossars einstellen
%\usepackage[style=super, header=none, border=none, number=none, cols=2, toc=true]{glossary}
%\renewcommand{\glossaryname}{Glossar}
%\makeglossary

% Source-Code Ausgabe
\usepackage{listings}
\definecolor{lstcolorkeyword}{rgb}{0,0,1}
\definecolor{lstcolorcomment}{rgb}{0,.501,0}
\definecolor{lstcolorstring}{rgb}{.843,0,0}
\lstset{
  language=XML, caption={}, captionpos=t,
  frame=single, frameround=ffff,
  basicstyle={\ttfamily\footnotesize},
  keywordstyle={\color{lstcolorkeyword}},
  commentstyle={\color{lstcolorcomment}\textit},
  stringstyle={\color{lstcolorstring}},
  identifierstyle={},
	backgroundcolor={},
	showspaces=false, showstringspaces=false, showtabs=false,
	tab=\rightarrowfill, tabsize=2, extendedchars=true,
	numbers=left, numberstyle=\scriptsize, stepnumber=1, numbersep=10pt,
	breaklines=true, breakautoindent=true, postbreak=\space,
	xleftmargin=20pt, xrightmargin=20pt,
	aboveskip=5pt, belowskip=12pt
}
\newskip\belowcaptionskipamount \smallskipamount=1pt
\renewcommand{\belowcaptionskip}{\belowcaptionskipamount}
\renewcommand{\lstlistingname}{Listing}
\renewcommand{\lstlistlistingname}{Listingverzeichnis}
\newcommand{\lstgetlistingref}[1]{\lstlistingname\ \ref{#1}}	% Eine Referenz zu Listing erhalten
% Bugfix
\makeatletter
	\renewcommand*{\lstlistoflistings}{%
	  \begingroup
	    \if@twocolumn
	      \@restonecoltrue\onecolumn
	    \else
	      \@restonecolfalse
	    \fi
	    \phantomsection
	    \addcontentsline{toc}{chapter}{\lstlistlistingname}
			\chapter*{\lstlistlistingname\@mkboth{\lstlistlistingname}{\lstlistlistingname}}		%\lol@heading
	    \setparsizes{\z@}{\z@}{\z@\@plus 1fil}\par@updaterelative
		\vskip0.4cm
	    \@starttoc{lol}%
	    \if@restonecol\twocolumn\fi
	  \endgroup
	}
\makeatother

% Zum Einbinden von Grafiken
\usepackage{graphicx}
\usepackage{floatflt}
\usepackage{wrapfig}
\newcommand{\setdefaultfiguresep}{\intextsep5pt}	% Abstand über und unter Figure
\newcommand{\setfloatfiguresep}{\intextsep2pt}
\newcommand{\setownfiguresep}[1]{\intextsep#1}
\setdefaultfiguresep
\abovecaptionskip2pt
\belowcaptionskip2pt
\setcapindent{1em}
%\setcapwidth[c]{160pt} setzt Breite für Caption
\addto\captionsngerman{\renewcommand{\figurename}{Abb.}}
\newcommand{\graphgetref}[1]{\figurename\ \ref{#1}}
\makeatletter
	\renewcommand{\listoffigures}{%
		\begingroup
			\if@twocolumn
				\@restonecoltrue\onecolumn
			\else
				\@restonecolfalse
			\fi
			\phantomsection
			\addcontentsline{toc}{chapter}{\listfigurename}
			\chapter*{\listfigurename\@mkboth{\listfigurename}{\listfigurename}} %\lof@heading
			\setparsizes{\z@}{\z@}{\z@\@plus 1fil}\par@updaterelative %\setparsizes{0}{0}{\z@\@plus 1fil}\par@updaterelative
			\@starttoc{lof}%
			\if@restonecol\twocolumn\fi
		\endgroup
	}
\makeatother

% Paket für Links innerhalb des PDF Dokuments
\usepackage{url}	% Zum Umbrechen langer URLs (keine Gewähr!!)
\usepackage[pdftex]{hyperref}
\hypersetup{
		pdftitle={\mypdftitle},
		pdfsubject={\mypdfsubject},
		pdfauthor={\mypdfauthor},
		pdfkeywords={\mypdfkeywords},
		%pdfcreator={},
		%pdfproducer={},
		%bookmarks,
		bookmarksopen,
		pdfstartview={FitH},	% Fit, FitH, FitV
		pdfpagelayout={SinglePage},
		pdfpagemode={UseOutlines}
}
\definecolor{LinkColor}{rgb}{0,0,0.5}
\hypersetup{colorlinks=true, linkcolor=LinkColor, citecolor=LinkColor,
		filecolor=LinkColor, menucolor=LinkColor, pagecolor=LinkColor, urlcolor=LinkColor,
		breaklinks=true
}

% Ein Paket, das die Darstellung von "Text, wie er eingegeben wird" erlaubt:
%  \begin{verbatim} \end{document}\end{verbatim} erzeugt die Ausgabe von
%  \end{document} im Typewrites-Style und beendet nicht das Dokument.
\usepackage{verbatim}
\usepackage{moreverb}

% Mit \blindtext kann Dummytext verwendet werden
%\usepackage{blindtext}

% Auflistungen erweitern
%\usepackage{paralist}
%\setlength{\leftmargin}{5pt} ???

% Itemize: Abstände verkleinern
\let\origitemize\itemize
\def\itemize{\origitemize\itemsep-2pt}
% Enumerate: Abstände verkleinern
\let\origenumerate\enumerate
\def\enumerate{\origenumerate\itemsep-3pt}
% Itemize / Enumerate: Einrückung verkleinern
\setlength\leftmargini{18pt}
\setlength\leftmarginii{7pt}