\documentclass[10pt,a4paper]{scrartcl}
\usepackage[ansi]{umlaute}
\usepackage{ngerman}
\usepackage{graphicx}
\usepackage[final]{listings}
\lstset{tabsize=2,basicstyle=\ttfamily\small}
\usepackage[draft]{fixme}
\usepackage{color}
\definecolor{LinkColor}{rgb}{0.0,0.0,0.0} 
\usepackage[
        bookmarks=true,
        bookmarksopen=true,
        bookmarksopenlevel={1},
        bookmarksnumbered=true,
        plainpages=false,
        pdfpagelabels=true,
        hypertexnames=false,
        pdftitle={},
        pdfauthor={Sven Wenzel},
        pdfcreator={LaTeX with hyperref and KOMA-Script},
        pdfsubject={},
        pdfkeywords={},
        final]{hyperref}
\hypersetup{colorlinks=true,
        anchorcolor=LinkColor,
        linkcolor=LinkColor,
        citecolor=LinkColor,
        filecolor=LinkColor,
        menucolor=LinkColor,
        pagecolor=LinkColor,
        urlcolor=LinkColor}

\newcommand{\hinweis}[1]{

\textbf{Hinweis:} #1

}

\title{SiDiff 2.0 -- Bundle-Overview}
\date{\today}
\author{Sven Wenzel}
\begin{document}
\maketitle
\tableofcontents
\newpage

\section{Introduction}
The SiDiff project has not been realized as a monolyth in a single Java project.
It rather consists of many subprojects. Each subproject is a OSGi bundle, which
provides Java classes and other resources. This document gives an overview over
the existing bundles.

We differentiate the following kinds of bundles:
\begin{description}
 \item [Applications] are applications on basis of the Eclipse OSGi implementation
\emph{Equinox}. Sie can read command line parameters and execute different services.
An example is provided by the UML Diff application, which compares two UML models.
 \item [Library Bundles] provide Java classes and resources that are used in most 
of the subprojects. They are not active, self-contained, or application-like components,
but they are only used for outsourcing recurring routines, such as filtering of
object sets.
 \item [Interface Bundles] define the components of SiDiff. Such components are realized
as OSGi service and has a clearly defined function (e.g. the management of similarities).
Hence, the interface bundle defines the interface of a service. Implementations are not 
part of such a bundle.
 \item [Implementation Bundles] provide the concrete implementation of a service. It
contains the code and the resources needed to realize a service (i.e. a component of
SiDiff). There can exist several implementation bundles for each interface bundle,
in that case each implementation provides a different realization. E.g. the storage
of matchings either in a table or as a graph.
 \item [Extension Bundles] usually contain sets of classes which are loaded reflectively.
They are used to extend the functionality of other components. An example is provided
by the compare functions, which are used to calculate the similarity between pairs of
model elements. With extension bundles the algorithmics are decoupled from documenttype-specific
functions.
 \item [Eclipse Plugins] are applications based on the Eclipse platform. They are 
complexer applications which usually contain a graphical user interface. Consequently,
they have many dependencies to Eclipse-specific libraries, which is the reason for
categorizing them seperately.
 \item [Test Bundles] are internal bundles, which are used for testing the different
SiDiff components.
\end{description}


\input{applicationlist.gen.tex}

\input{librarylist.gen.tex}

\input{interfacelist.gen.tex}

\input{implementationlist.gen.tex}

\input{extendlist.gen.tex}

\input{eclipselist.gen.tex}

\input{testslist.gen.tex}

\input{wronglist.gen.tex}

\input{undocumentedlist.gen.tex}

\end{document}
