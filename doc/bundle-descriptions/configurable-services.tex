Konfigurierbare Services wie z.B. Annotation, Matching, etc. werden ueber sogenannte ServiceManager verwaltet.

Beispiel:
-----------------------------------------------------------------------
public class Activator implements BundleActivator {

	FunctionServiceManager reg;
	/*
	 * (non-Javadoc)
	 * @see org.osgi.framework.BundleActivator#start(org.osgi.framework.BundleContext)
	 */
	public void start(BundleContext context) throws Exception {
		reg = new FunctionServiceManager(context);
		context.registerService(new String[]{FunctionServiceManager.class.getName()}, reg, null);
	}

	/*
	 * (non-Javadoc)
	 * @see org.osgi.framework.BundleActivator#stop(org.osgi.framework.BundleContext)
	 */
	public void stop(BundleContext context) throws Exception {
		reg.unregister(null);
	}
}

-----------------------------------------------------------------------


public class FunctionServiceManager {
	
	private BundleContext context;
	
	public FunctionServiceManager(BundleContext context) {
		this.context = context;
	}

	public void register(String variant, String configdata) {
		// konfigurieren
		Function function = new Function();
		function.config(variant, configdata);
		// registrieren
		context.registerService(Function.class.getName(), function, function.getProperties());
	}
	
	public void unregister(String filter) throws InvalidSyntaxException {
		ServiceReference[] refs = context.getAllServiceReferences(Function.class.getName(), filter);
		for (ServiceReference reference : refs)
			context.ungetService(reference);
	}
}



